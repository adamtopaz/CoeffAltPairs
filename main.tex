\documentclass[12pt]{amsart}

\title{Alternating pairs with coefficients}
\author{Adam Topaz}
\address{
    Mathematical and Statistical Sciences \\
    University of Alberta \\
    632 Central Academic Building \\
    Edmonton AB T6G 2G1 \\
    Canada }
\subjclass{12G99, 12J10, 19D45}
\email{topaz@ualberta.ca}
\urladdr{https://adamtopaz.com/}
\date{\today}
\thanks{Research supported by the author's NSERC discovery grant.}
\keywords{Alternating pairs, valuation theory, anabelian geometry}
\date{Last updated: \today}

\usepackage{amssymb}
\usepackage{hyperref}
\usepackage{microtype}

\newcommand{\HH}{\operatorname{H}}
\newcommand{\Qbb}{\mathbb{Q}}
\newcommand{\Cbb}{\mathbb{C}}
\newcommand{\Hom}{\operatorname{Hom}}
\newcommand{\Dcal}{\mathcal{D}}
\newcommand{\Ical}{\mathcal{I}}
\newcommand{\smin}{\smallsetminus}
\newcommand{\Pbb}{\mathbb{P}}
\newcommand{\Abb}{\mathbb{A}}
\newcommand{\Zbb}{\mathbb{Z}}
\newcommand{\mfrak}{\mathfrak{m}}
\newcommand{\Ocal}{\mathcal{O}}
\newcommand{\Hcal}{\mathcal{H}}
\renewcommand{\tilde}{\widetilde}

\newtheorem{maintheorem}{Theorem}
\renewcommand*{\themaintheorem}{\Alph{maintheorem}}

\begin{document}

\maketitle

\begin{abstract}
  We prove a variant of the fundamental theorem of alternating pairs which works for arbitrary fields of positive characteristic $p$ and arbitrary coefficient fields of characteristic not dividing $2 \cdot p$.
\end{abstract}

\tableofcontents

\section{Introduction}

Most results in anabelian geometry have a \emph{local step} where the goal is to detect local information using the given data, which is usually of Galois-theoretic or cohomological nature.
In birational contexts, this local data usually refers to structures arising from \emph{valuations} on the field(s) in question.

Recently, these techniques have been applied by the author in \emph{motivic} contexts.
For example, in~\cite{https://doi.org/10.48550/arxiv.1705.01084} it was shown that the \emph{generic Betti cohomology ring} $\HH^{*}(K|k,\Qbb(*))$ of a higher-dimensional function field over an algebraically closed subfield $k \subset \Cbb$, endowed with its natural mixed Hodge structure in degree one, completely determines the fields $K|k$.
An \emph{integral} $\ell$-adic analogue of this was also obtained in~\cite{https://doi.org/10.48550/arxiv.1910.03563}.

In both of these cases, detecting the valuations of the function field in question was the key first step, and the essential tool used to accomplish this was the so-called \emph{fundamental theorem of alternating pairs}.
Variants of this result have appeared several times in the literature (albeit with slightly different names), in the work of Bogomolov~\cite{zbMATH00124400}, Bogomolov-Tschinkel~\cite{zbMATH02078165}, Efrat~\cite{zbMATH01489500}, Koenigsmann~\cite{zbMATH00785379}, Engler-Koenigsmann~\cite{zbMATH01135369}, and the author~\cite{zbMATH06778776,https://doi.org/10.48550/arxiv.1705.01084}.

However, the one thing that all of these results have in common is that the coefficient ring used must, in some sense, be \emph{prime}.
This is well-suited for contexts arising from Galois theory, where one can essentially work with Galois cohomology $\HH^{*}(K,\Zbb/\ell^{n})$ or $\HH^{*}(K,\Zbb_{\ell})$.
Similarly, in the context of generic Betti-cohomology $\HH^{*}(K|k,\Qbb(*))$ mentioned above, the coefficient ring is $\Qbb$ and thus the results apply.

On the other hand, in order to obtain a more complete \emph{motivic} picture, it is desirable to consider objects constructed using \emph{crystalline cohomology}, or, since we wish to eventually work with nonproper varieties, using rather \emph{rigid cohomology}.
The function fields in question here are regular finitely generated fields over perfect fields $k$ of positive characteristic $p$, while the coefficient ring is the fraction field $F$ of the $p$-typical Witt vectors $W(k)$ of $k$.
Unfortunately, such an $F$ is usually far from being prime and thus the results used above no longer apply directly!

In this short note, we overcome this hurdle by proving a variant of the fundamental theorem of alternating pairs which is applicable to contexts arising from rigid cohomology as discussed above.
We do intend to apply this result as an initial step in studying the anabelian properties of generic rigid cohomology.
However, the theorem we present here involves only elementary objects and may be of independent interest, so we defer the discussion of these applications to future work.

Our main result is formulated as follows.

\begin{maintheorem}\label{maintheorem}
  Let $K$ and $F$ be two fields.
  Assume that $K$ has positive characteristic $p$, that the characteristic of $F$ does not divide $2 \cdot p$.
  Let $\Dcal$ be a subspace of $\Hom(K^{\times},F)$, which is closed in the weak topology, where $F$ is considered as being discrete.
  Assume that for all $f,g \in \Dcal$, and all $x,y \in K^{\times}$ such that $x + y = 1$, one has
  \[ f(x) \cdot g(y) = f(y) \cdot g(x). \]
  Then there exists a valuation $v$ of $K$ such that one has
  \[ \Dcal \subset \Hom(K^{\times}/(1+\mfrak_{v}),F) \subset \Hom(K^{\times},F) \]
  and, setting $\Ical := \Hom(K^{\times}/\Ocal_{v}^{\times},F) \subset \Hom(K^{\times},F)$, the intersection $\Dcal \cap \Ical$ in $\Hom(K^{\times},F)$ has codimension $\le 1$ in $\Dcal$.
\end{maintheorem}

The main novelty in this theorem, at least in comparison with its predecessors, is in the rather weak assumptions on the field $F$.
However, this comes at the expense of imposing an additional assumption on $K$.
The proof of this result stemmed from a detailed investigation of the \emph{rational case}, as described in~\cite[\S A]{https://doi.org/10.48550/arxiv.1705.01084}.

In fact, the proof of Theorem~\ref{maintheorem} has been formally verified using the \texttt{Lean3} interactive theorem prover~\cite{10.1007/978-3-319-21401-6_26} and its mathematics library \texttt{mathlib}~\cite{10.1145/3372885.3373824}.
The formally verified proof of Theorem~\ref{maintheorem} can be found in~\cite{lean-acl-pairs}, specifically in the file \href{https://github.com/adamtopaz/lean-acl-pairs/blob/6ac31d86ca2739b6c18d3f05b7007e720f66299f/src/main_theorem_char.lean}{\texttt{src/main\_theorem\_char.lean}} from commit \href{https://github.com/adamtopaz/lean-acl-pairs/tree/6ac31d86ca2739b6c18d3f05b7007e720f66299f}{\texttt{6ac31d8}} which is the most recent one on the master branch at the time this note was written.
The converse of this theorem is valid in full generality with no further assumptions on $K$ and/or $F$, and a formal proof of this converse can also be found in the repository mentioned above.

We give an indication of the argument in the next section, explaining only the key ideas.
We refer the reader to the formally verified proof for all the remaining details.

\section{A sketch of the proof}

The argument proceeds similarly to the proof of~\cite[Theorem~A.3]{https://doi.org/10.48550/arxiv.1705.01084}.
Namely, we proceed by contradiction, which is obtained by carrying out some explicit calculations in a certain projective space.
The argument here differs precisely at the point where one obtains the desired contradiction.

Let $\Dcal$ be as in the statement of the theorem and put $T := \Dcal^{\perp}$, the orthogonal of $\Dcal$ with respect to the pairing
\[ K^{\times} \times \Hom(K^{\times},F) \to F. \]
Let $H$ denote the subgroup of $K^{\times}$ which is generated by $T$ and all $x \in K^{\times} \smin H$ such that $1 + x \notin T \cup x \cdot T$.
By~\cite[Theorem 2.16]{zbMATH04023378}, there exists a valuation $v$ of $K$ and a subgroup $\tilde H \subset K^{\times}$ containing $H$, such that $1 + \mfrak_{v} \subset T$, $\Ocal_{v}^{\times} \subset \tilde H$ and $[\tilde H : H] \le 2$.

Letting $\Hcal \subset \Hom(K^{\times},F)$ denote the dual of $K^{\times}/\tilde H$, we must therefore show that $\Dcal/\Hcal$ has dimension $\le 1$.
By the construction of $H$ and using the fact that $F$ has characteristic $\neq 2$, it suffices to show that for all $\Phi : K^{\times} \to F \times F$ of the form $\Phi = (f,g)$ for $f,g \in \Dcal$, and all $x,y \in K \smin \{0,-1\}$ such that $\Phi(1+x) \notin \{\Phi(1),\Phi(x)\}$ and $\Phi(1+y) \notin \{\Phi(1),\Phi(y)\}$, the pair $\Phi(x)$ and $\Phi(y)$ is \emph{linearly dependent} over $F$.

Assume this is not the case, and let $\Phi = (f,g)$, $f,g \in \Dcal$ and $x,y$ witness this.
Thus, $\Phi(x)$ and $\Phi(y)$ are \emph{linearly independent} while $\Phi(1+x) \notin \{\Phi(1),\Phi(x)\}$ and $\Phi(y) \notin \{\Phi(1),\Phi(y)\}$.
Note that the condition on $\Dcal$ ensures that $\Phi(a+b),\Phi(a),\Phi(b)$ are colinear (in the affine sense) whenever $a,b,a+b \in K^{\times}$.

Embed $F \times F = \Abb^{2}(F)$ into $\Pbb^{2}(F)$ in the usual way, and compose
\[ K^{\times} \xrightarrow{\Phi} F\times F = \Abb^{2}(F) \hookrightarrow \Pbb^{2}(F) \]
with an projective-$F$-linear automorphism of $\Pbb^{2}(F)$ to obtain
\[ \Psi : K^{\times} \to \Pbb^{2}(F) \]
satisfying the following conditions (using homogeneous coordinates):
\begin{enumerate}
  \item $\Psi(1) = (1:0:0)$; $\Psi(x) = (1:1:0)$; $\Psi(y) = (1:0:1)$.
  \item $\Psi(1+x) = (0:1:0)$; $\Psi(1+y) = (0:0:1)$.
  \item $\Psi(a+b),\Psi(a),\Psi(b)$ are colinear whenever $a,b,a+b \in K^{\times}$.
\end{enumerate}
The crux of the argument is to show that for all natural numbers $m$, one has
\begin{align}\label{keyeqn}
\Psi(m + (m+1) \cdot x) = (1:(m+1):0).
\end{align}
Here one argues similarly to Step~4 in the proof of~\cite[Theorem A.3]{https://doi.org/10.48550/arxiv.1705.01084}.
This suffices to obtain our desired contradiction by taking $m = p$ to be the characteristic of $K$, since that would imply that
\[ (1:1:0) = \Psi(x) = (p + (p+1) \cdot x) = (1:p+1:0) \]
hence $p = 0$ in $F$, which contradicts the hypothesis on $F$.
Again, we remark that this is precisely the point where the our argument differs from that of~\cite[Theorem A.3]{https://doi.org/10.48550/arxiv.1705.01084}, where a contradiction is obtained in a different way.

Here is a sketch of the proof of~\eqref{keyeqn}.
First, use the fact that
\[ 1+x+y = (1+x) + y = (1+y) + x \]
along with condition (3) to see that $\Psi(1+x+y) = (1:1:1)$.
Next, use the fact that
\[ 2+x+y = (1+x) + (1+y) = 1+(1+x+y) \]
along with condition (3) to see that $\Psi(2+x+y) = (0:1:1)$.
Finally, we show by induction on $m$ that
\[ \Psi((m+1)+(m+1) \cdot x + y) = (1:m+1:1) \]
and
\[ \Psi(m+(m+1) \cdot x) = (1:m+1:0). \]

The base case $m = 0$ follows from the above.
For the inductive case, we use condition (3), the inductive hypothesis and the equation
\begin{align*}
  (m+2)+(m+2) \cdot x + y &= (m+(m+1) \cdot x + y)+(2+x+y)  \\
                          &= ((m+1)+(m+1) \cdot x + y) + (1+x)
\end{align*}
to find $\Psi((m+2)+(m+2) \cdot x + y) = (1:m+2:1)$.
Conclude by using condition (3), the calculation above, the inductive hypothesis, and the equation
\begin{align*}
  (m+1) + (m+2) \cdot x &= ((m+2)+(m+2) \cdot x + y) - (1+y) \\
                        &= (m + (m+1) \cdot x) + (1 + x).
\end{align*}

\bibliographystyle{amsplain}
\bibliography{refs}

\end{document}
